\documentclass[12pt]{report}

\usepackage{polyglossia}
\setdefaultlanguage{english}
\setotherlanguage[variant=ancient]{greek}

%\usepackage[style=philosophy-classic, language=english, uniquename=init, mincitenames=2, minbibnames=2]{biblatex}
%\addbibresource{/home/user/.local/share/latex/general.bib}

\title{Climate change and energy transition}
\date{Autumn 2023}
\author{Wester, T.I.}


\renewcommand{\chaptername}{Lecture}


\begin{document}
\maketitle
\tableofcontents

\chapter{}

A central component to climate change is carbon dioxide, which is highly related
to our energy use. Transitioning to a climate friendly global lifestyle, we must
keep in mind the wish to keep our energy budget stable. Energy availability is
tightly correlated with both economic growth and Human Development Index,
neither of which are generally desired to reduce.

When discussing climate problems, there is an intrinsic need for an
interdisciplinary approach. There are for instance questions raised by science
concerning whether we have entered a new geological epoch, how we might get to a
stable and sustainable system regarding climate, and how we might bring justice
to those who greatly damage the climate. However, each of these questions have
normative-, psychological-,  and sociological- dimensions.

\section{The Anthropocene}

The concept of the Anthropocene, the age of man, wherein humans have the
greatest control over the earth, is highly present in the climate change debate.
While the question: ``Have we entered the Anthropocene?'' is a matter of natural
science, the attitude towards it is a philosophical matter.

There are those who fully embrace the Anthropocene, and wish therefore to take
responsibility for the maintenance of the earth.

This approach stands in great contrast one wherein the Anthropocene is despised,
but seen nevertheless as inevitable. This approach typically does not lead to
taking responsibility.

Belatedly, but more responsible, are those who wish to limit the Anthropocene,
mitigating and reducing the negative impact humans have.

\bigskip

There are also other terms proposed against the term ``Anthropocene'', for
instance Homogenocene or Eromocene. Respectively, the age of homogenisation
and the age of loneliness (both referring to biodiversity loss).

\section{Sustainability}

Sustainable development is defined by the WCED as development that meets the
needs of the present without comprising the ability of future generations to
meet their own needs. This definition limits itself greatly to human ability,
and does not properly define the needs current-, and future- generations have.

The assumption is that this ``sustainability'' is indeed possible, partly
because the ethical and political implications of not meeting the needs to the
present are too-repugnant to consider. This belief is not based in empirical
evidence, but rather a matter of necessary hope.

\end{document}
