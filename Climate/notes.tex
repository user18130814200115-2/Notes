\documentclass[12pt]{report}

\usepackage{polyglossia}
\setdefaultlanguage{english}
\setotherlanguage[variant=ancient]{greek}

%\usepackage[style=philosophy-classic, language=english, uniquename=init, mincitenames=2, minbibnames=2]{biblatex}
%\addbibresource{/home/user/.local/share/latex/general.bib}

\title{Climate change and energy transition}
\date{Autumn 2023}
\author{Wester, T.I.}


\renewcommand{\chaptername}{Lecture}


\begin{document}
\maketitle
\tableofcontents

\chapter{}

A central component to climate change is carbon dioxide, which is highly related
to our energy use. Transitioning to a climate friendly global lifestyle, we must
keep in mind the wish to keep our energy budget stable. Energy availability is
tightly correlated with both economic growth and Human Development Index,
neither of which are generally desired to reduce.

When discussing climate problems, there is an intrinsic need for an
interdisciplinary approach. There are for instance questions raised by science
concerning whether we have entered a new geological epoch, how we might get to a
stable and sustainable system regarding climate, and how we might bring justice
to those who greatly damage the climate. However, each of these questions have
normative-, psychological-,  and sociological- dimensions.

\section{The Anthropocene}

The concept of the Anthropocene, the age of man, wherein humans have the
greatest control over the earth, is highly present in the climate change debate.
While the question: ``Have we entered the Anthropocene?'' is a matter of natural
science, the attitude towards it is a philosophical matter.

There are those who fully embrace the Anthropocene, and wish therefore to take
responsibility for the maintenance of the earth.

This approach stands in great contrast one wherein the Anthropocene is despised,
but seen nevertheless as inevitable. This approach typically does not lead to
taking responsibility.

Belatedly, but more responsible, are those who wish to limit the Anthropocene,
mitigating and reducing the negative impact humans have.

\bigskip

There are also other terms proposed against the term ``Anthropocene'', for
instance Homogenocene or Eromocene. Respectively, the age of homogenisation
and the age of loneliness (both referring to biodiversity loss).

\section{Sustainability}

Sustainable development is defined by the WCED as development that meets the
needs of the present without comprising the ability of future generations to
meet their own needs. This definition limits itself greatly to human ability,
and does not properly define the needs current-, and future- generations have.

The assumption is that this ``sustainability'' is indeed possible, partly
because the ethical and political implications of not meeting the needs to the
present are too-repugnant to consider. This belief is not based in empirical
evidence, but rather a matter of necessary hope.

\chapter{Values and Climate Models}

There is a certain level of uncertainty regarding climate change. There are
various details for which conflicting evidence exists. We do know however that
we are not altogether wrong about anthropocentric climate change. Climate
sceptics almost never pass even the lowest bar for counting as scientific, but
then again, so do many historically-fringe opinions which we now count as fact.

While we know that climate change is largely-, if not primarily- caused by the
emission of greenhouse gasses. The extent to which each affects the global rise
in temperature, is not entirely known. Much like, local effects as opposed to
global ones, various chemical reactions, threshold-theories, and various other
facts.

One may claim that we are to simply adhere to those facts that are exposed to us
through use of the scientific method. However, there are various scientific
methods, whose exact parameters are not rigid. Among scientific models there are
inductive-, deductive-, and abductive- ones, each with their own threshold for
``truth''.

\section{Models}

A large part of the scientific study of climate change is climate modeling. The
aim of a model is to simplify a given environment (called ``the target system'')
to predict the impact of various changes to said environment. Climate models in
particular often combine various models, such as atmospheric-, tectonic-, and
glacial- models, and various complex interactions between them.

We use our models to predict outcomes and inform action, though, due to the fact
that models are -- by definition -- simplified, the modeled outcome rarely
\emph{entirely} conforms to reality.

An ``ensemble model'' is a model which combines the average output of various
models which are calibrated for slightly different purposes. With an ensemble
model, we can project how uncertain we are about a given outcome by analysing
the divergence of each model to the ensemble model.  Using an ensemble model
however relies on certain assumptions, such as all models being equally good in
measuring, and that each of them are scattered around the ``true'' value.

While there are good reasons not to use ensemble models (such a rejecting one of
the assumptions made for using such models), they do show an important matter of
philosophy of science, that being that no hypothesis is every entirely certain.
Meaning that scientists, in accepting theories, make certain value judgements
which lie outside the realm of hard-science.

There is for instance the matter of tolerating false negatives and false
positives. In the case of the climate discussion, false negatives are often
extremely catastrophic, whereas false positives are a matter of wasted funds.
This might lead one to tolerate false positives but not false negatives (or
rather a probability of a given negative being false), but this is a value
judgement which ought to be made by a greater community than merely the
scientific one.

Such a case is argued by Jeffrey, who states that scientists should only express
their expert judgement, not their descriptive claims. Who then is left to make
descriptive claims? This question is not answered to Jeffrey, though a
compelling case can be made to not merely leave this up to the public, nor to a
group working entirely separately from the scientists publishing the
probabilistic facts.

Winsberg argues against Jeffrey and states that scientists can make value
judgements \emph{if} they are transparent about their values. To what extent we
\emph{can} be transparent though is not entirely clear. The value-leadenness of
a model lies in the details, not all of which are obvious to the scientists
themselves, let alone the public,

\chapter{Climate Economics}
\section{Social Discount Rates}
In the case of averting and reverting climate change, the question of cost
dispersal is a big topic. The costs of reversion are high, and the potential
agents capable-, and obliged-, to pay are spatially and temporally dispersed, as
are the victims. Temporal dispersion is particularly problematic, as there is no
straightforward way to assign \emph{current value} to future generations.
Economists have formalized this problem as $\delta + \eta \cdot \gamma = \rho$
where
\begin{tabular}{rcl}
\delta & = & Pure time discount rate\\
\eta & = & Marginal increase of utility per unit of wealth\\
\gamma & = & Elasticity of wealth growth\\
\rho & = & Social discount rate
\end{tabular}

\subsection{Pure time discount rates}
Using this way of calculating the social discount rate, we encounter problems,
largely because this method was intended for economics. For instance, with a
social discount rate of 5\%, one death today would equal a billion deaths 400
years from now, and 5\% is a rather modest discount rate (in the intended
economic context). However, giving no social discount rate, the threat of
eventual extinction -- which, no matter how far off, seems currently inevitable
-- would overrule all current considerations.

In the discussion of social discount rates, there is a distinction to be made
between pure time discounting, and growth rate discounting, with the labels
speaking for themselves.

In the case of time discount rates, when applied to moral considerations, the
question is asked in what way one's position in time is relevant to moral
status. These arguments are particularly popular with utilitarians. In objection
is said that having no time discount rate is unduly demanding for current
generations, as it requires too many sacrifices on our account. To solve this,
utilitarians introduced growth discounting.

\subsection{Growth discount rate}
Since future generations will have more wealth, we ought not pay for them, for
it unjust for the poor to pay the expenses of the rich. This makes it such that
current generations do not need to make great sacrifices, whist still
maintaining that one's position in time is irrelevant to moral consideration.

While this principle on this own allows us to delay action, there are further
considerations which make such delays problematic. It is for instance uncertain
weather certain problem can still be solved in the future, weather future
generations will be wealthier (in every sens of the world, monetarily,
demographically, ecologically, etc.)

Rather than delaying action, one may instead argue for deferring costs. This
would imply that we take action now, but push the cost of action to future
generations. The amount of cost that may be deferred is dependent on the
discount rate, and the damage caused by inaction, and non-deferral. Furthermore,
not deferring costs is politically unrealistic.

\end{document}
