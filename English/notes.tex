\documentclass[12pt]{report}

\usepackage{polyglossia}
\setdefaultlanguage{english}
\setotherlanguage[variant=ancient]{greek}

%\usepackage[style=philosophy-classic, language=english, uniquename=init, mincitenames=2, minbibnames=2]{biblatex}
%\addbibresource{/home/user/.local/share/latex/general.bib}

\title{Middle English}
\date{Autumn 2023}
\author{Wester, T.I.}


\renewcommand{\chaptername}{Lecture}


\begin{document}
\maketitle
\tableofcontents

\chapter{}

\textit{Corpus distributed in class}

The primary course book will be "an introduction to middle English".

Middle English (defined as the period between the Norman conquest and the
introduction of the printing press) is handed down to us entirely in
hand-written manuscript of which several copies tend to exist.

Hunter.uma.es

The Hengwrt is the oldest preserved manuscript of the Canterbury
tales\footnote{Known in old manuscripts as ``The book of the tales of
Canterbury''}. The manuscript is written in \textit{Gothic Cursiva Anglicana
Formata}. This script is part of the Gothic family (the exclusive family after
the Norman conquest).  Of this family there are two types, one of which is
\textit{Anglicana}.  The script is written cursively though adapted for
literature (though separation of the letters) (indicated by the \textit{Formata}
in the script name).

Common in old manuscripts is the concept of biting, which is where the two
letters share a side, such as in \ae.

Modern elements of English absent from Middle English include the term ``who''
as a relative pronoun, being substituted instead by ``that''. Similarly,
do-support for verbs has not yet come into being.



\end{document}
