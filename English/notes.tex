\documentclass[12pt]{report}

\usepackage{polyglossia}
\setdefaultlanguage{english}
\setotherlanguage[variant=ancient]{greek}

%\usepackage[style=philosophy-classic, language=english, uniquename=init, mincitenames=2, minbibnames=2]{biblatex}
%\addbibresource{/home/user/.local/share/latex/general.bib}

\title{Middle English}
\date{Autumn 2023}
\author{Wester, T.I.}


\renewcommand{\chaptername}{Lecture}

\newcommand{\eth}{ð}
\newcommand{\thorn}{þ}
\newcommand{\wynn}{ƿ}
\newcommand{\yogh}{ȝ}

\begin{document}
\maketitle
\tableofcontents

\chapter{}

\textit{Corpus and Various documents distributed in class}

The primary course book will be "an introduction to middle English".

Middle English (defined as the period between the Norman conquest and the
introduction of the printing press) is handed down to us entirely in
hand-written manuscript of which several copies tend to exist.

Hunter.uma.es

The Hengwrt is the oldest preserved manuscript of the Canterbury
tales\footnote{Known in old manuscripts as ``The book of the tales of
Canterbury''}. The manuscript is written in \textit{Gothic Cursiva Anglicana
Formata}. This script is part of the Gothic family (the exclusive family after
the Norman conquest).  Of this family there are two types, one of which is
\textit{Anglicana}.  The script is written cursively though adapted for
literature (though separation of the letters) (indicated by the \textit{Formata}
in the script name).

Common in old manuscripts is the concept of biting, which is where the two
letters share a side, such as in \ae.

Modern elements of English absent from Middle English include the term ``who''
as a relative pronoun, being substituted instead by ``that''. Similarly,
do-support for verbs has not yet come into being.

\section{Lecture}

\textit{Lecture slides supplied on-paper in class and available online}

English arrives in Britain from the continent (Friesland, Germany, and Saxony)
after the departure of the Romans. At this time, most people speak a Germanic
language close to Welsh, pushing away the indigenous Celts who feel to the
Scottish Ilses, Cornwall and Ireland.

As the viking raids start in Northumbria (793), there is more and more contact
between the English west-Germanic language, and Scandinavian north-Germanic
language. The vikings, other than raiding, likely settled peacefully coexisting
with the Anglo Saxons. Given this extended coexistence, there is a great share
of vocabulary which arises only in manuscripts of Middle English. Old English
manuscripts survive primarily out of areas such as Wessex, which had little
Scandinavian (Danish) influence. Towards the end of this period -- right before
the Norman conquest -- a Norwegian sits on the throne, leaning -- after his
death -- to a dispute between the Norwegians, English, and French.

After the Norman conquest, the official language of England became French with
all government officials, aristocrats, and bishops being French nationals. This
conquest affects English in many ways besides language, such as format,
handwriting and poetry. After this period, when English returns as a written
language, no one knows how to spell it, having written only French and Latin for
300 years, leading to great variation in spelling.

\chapter{Codicology and Paleography}

\textit{Lecture slides supplied on-paper in class and available online}

Middle English (and in fact, all medieval manuscripts) contain inherent
ambiguity due to the difficulty of recognizing handwriting, with manuscripts
containing otiose marks\footnote{Serving no linguistic purpose}, having no
normalized spacing or capitalisation, and frequent use of abbreviation.

A letter can defined by its nomen, figura, and potestas, which together from the
littera.

\begin{tabular}{l|l}
Term & Meaning \\
\hline
	Littera & Letter \\
	Nomen & Name \\
	Figura & Shape \\
	Potestas & Sound (literally ``power'')\\
\end{tabular}

\chapter{Manuscript characteristics}

\textit{Lecture slides supplied on-paper in class and available online}

In comparison to Old English, Middle English is lends itself much better to
sociolinguistic analysis.

A major advantage for Middle English is the amount of
texts that were preserved. In particular this allows us to separate Middle
English characteristics from the idiosyncrasies from a particular scribe. This
is aided particularly by the fact that we have the same manuscript copied by
different scribes (such as 80 copies of the Canterbury tales), and different
manuscripts by the same scribe (4 manuscripts in the case of Chaucer for
instance).

A secondary element of Middle English which is entirely absent from Old English 
is the presence of plays and court transcripts, which allow for the
reconstruction of actual spoken language, as opposed to mere poetic
representation. A similar reconstruction can be formed from other works such as
the Canterbury tales due to the presence of dialogues.

Neither the Old-, nor Middle- English, corpus has many authorial copies. This is
likely because authors wrote on wax tablets, passing these on to scribes for
copying, and subsequently meting down the wax for reusing.

Middle English also contains the first explicit statement of differences in
dialect in comments, as well as attempts at recreating the forms of speech of
different dialects. Of course, there is also dialectical difference among
manuscripts themselves, which was present in writing likely as much as it was in
speech.

\chapter{Important Middle English Manuscripts}

Old English is mostly written in alliterative verse, as is common in Germanic
languages, whereas middle English -- influenced by French -- has end-rime verse.
In later Middle English however, there is a revival of alliterative rime. This
revival is occasionally accompanied with intentionally archaic hand and
spelling.

End-rime verse is popularized by secular scribes in London around 1400. Chaucer
particularly popularizes verses with ten syllables and five stresses per line.
Scribes with less French influence -- such as those located in the Southwest
Midlands, East Anglia and Yorkshire -- stick more to the Old English traditions.
Thus, many of the English texts written in these areas are in Alliterative
meter, though there were also many written in French and Latin, occasionally
within the same manuscript.

Interestingly, there are scribes from these areas, who write rather modern
Latin, though they use many old English idiosyncrasies such as split
ascenders, as well as the \eth and \wynn.

In Anglo-Saxon times, the layout of a manuscript is not particularly important,
though this becomes more important as the middle ages progress. Of course, for
proper \emph{ordinatio} -- or page management -- takes much more work and effort
compared to the simpler-, business-inspired- manuscripts of Old English.

One example of \emph{ordinatio} is Harley 2253 $70^v$. In this two-column text,
the religious texts start higher on the page than the secular texts. A further
popular layout in Middle English is bob-and-wheel style printing. Where two
lines rime, with a third (located vertically between, horizontally on the next
column) comments on both previous lines.

\chapter{Standardisation}

Standardisation, or the reduction of the variations in spelling, grammar, and
pronunciation, already happened in late west Anglo-Saxon before the Norman
Conquest. After the Norman conquest, French took over administration, thereby
robbing English of some of its functions and thereby setting back the project of
standardisation. As English becomes more common in parliament, schools, and
administration, English continues to standardise.

Throughout the history of Middle English, there are four streams converging
towards standardisation.\\
\begin{tabular}{l|l}
Central Midlands & Religious, associated with Wycliffe and the Lollards\\
Earth 14th century London & Edinbrugh and the national library of Scotland\\
Later 14th centry London & Chaucer and national parliament (10\%)\\
Post 1430 & Westminster and national parliament
\end{tabular}

Standardisation generally goes through four stages: Acceptance, Selection,
Codification, and Elaboration. The above, ``standards'' did not go through all
stages, particularly codification and elaboration are missing. English does not
go through all of these stages until the use of the printing press becomes
widespread.

Another problem for the ``standards'' listed above, is that there is a limited
number of scribes, genres, and texts for each standard. Among these standard
there are some which are only written by one or two scribes, and only in
end-rime verse poetry. Furthermore, there are manuscripts by multiple scribes,
some of whom write in one of the standards above, whereas others do not,
irrespective of communication between said scribes.

\chapter{}

The Trinity Gower is a manuscript of John Gower's \textit{Confessio Amantis}
copied by five scribes.

\begin{tabular}{l|l}
	Scribe & Manuscripts\\
	A & Not known for other manuscripts\\
	B & Also copied Hengwrt and Ellesmere\\
	C & Not known for other manuscripts\\
	D & Also copied Corpus and Harley\\
	E & Thomas Hoccleve\\
\end{tabular}

Scribe B may have been Chaucer's own scribe, while Thomas Hoccleve may have been
Gower's own scribe.  All of the scribes were present in London-Westminster
around the same time and appear to have worked rather separately. Their copying
is not particularly organized, leading to much of the variation we see.

While the area in which the scribes worked can rightly be called ``London'', the
term ``Westminster'' might be more accurate and specific. Historically, London
was seen as the center of literary texts, while Westminster was the center of
documents. We now know that there is a lot of overlap between these groups. Such
as is the case with Hoccleve, whose Westminster training is evident, though he
neither worked nor lived at Westminster. Furthermore, he also copied literary
(London-style) texts.

\chapter{Dialectology}

The Principal work in middle English Dialectology is the Linguistic Atlas of
(Late/Early) Middle English (respectively LALME and LAEME). LALME is based on
scribal translations (localisations) of original manuscripts. These
translations are dialectically consistent and can therefore be used for
dialectical analysis.

\chapter{Provincial book production}

\section{Auchinleck}
The Auchinleck manuscript is a rather peculiar one, containing 44 texts --
exclusively in English, mostly romances, many of which did not appear in earlier
manuscripts (17 to be precise). Furthermore, these stories are tied together by
characters and entire passages making appearances between texts in the same
manuscript.

The manuscript was commercially produced in London by 5-6 professional scribes.
We know of its London origins because of the novelty of the texts, as it would
not be possible elsewhere to have aces to such texts and aces to translators. It
is furthermore theorised, due to the crossovers, that the scribes worked under
one roof. Though these echoes are likely merely a characteristic of the genre,
as professional scriptoria did not exist.

The manuscript emerged really too-early for English to become the aristocratic-,
and legal- language. As such, the manuscript was likely produced for a merchant,
banker, or other wealthy artisan. This is furthermore supported by the somewhat
sparse decorations, though the product was likely still very expensive. Sue to
the expenses, it is also theorised that the manuscript was produced by merchants
or bank notaries themselves.  The key text in the manuscript speaks about the
Guy of Warwick, which leads some to theorise the Earls of Warwick to be the
patrons, though they would not be speaking English. Though the text may be
written for the children of Warwick to educate them in morals and English
simultaneously.

\section{Thornron} Thornton was a rural gentleman in Yorkshire who spent 40-50
year assembling 2 manuscripts. Thornton was not a professional scribe, as is
recognizable by his handwriting and quire management. The manuscripts were
likely intended to a family library as it is of highly varied content created by
a single scribe. It furthermore contains gemological records on the blanks of
the quires. The family has also recorded such gemological records after
Thornton's death in the same book.

How Thornton got a hold of the exemplars remains a question open to scholarship.
The most prolific theory being a local network of book-sharing, both religious
and secular.

Thornton furthermore stands out because he signed all of his texts. This is
likely part to credit his accomplishment, though it may have also involved
indication of ownership in the aforementioned sharing networks.

\section{Findern} The Findern manuscript originates from Derbyshire Findern,
though is possible unconnected to the Findern family. This manuscript is copied
by several scribes (up to 50 individuals), even on the same page. This leads
some to conclude that this manuscript may have been part of a project of family
entertainment. The hands furthermore range between amateur and professional.
Many of the scribes signed their sections, whose signatures included women's
names. These women's hands were furthermore more inclined towards the
professional side of the spectrum. For this manuscript too, local sharing
networks are proposed as a source of the exemplars.

\end{document}
