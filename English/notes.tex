\documentclass[12pt]{report}

\usepackage{polyglossia}
\setdefaultlanguage{english}
\setotherlanguage[variant=ancient]{greek}

%\usepackage[style=philosophy-classic, language=english, uniquename=init, mincitenames=2, minbibnames=2]{biblatex}
%\addbibresource{/home/user/.local/share/latex/general.bib}

\title{Middle English}
\date{Autumn 2023}
\author{Wester, T.I.}


\renewcommand{\chaptername}{Lecture}

\newcommand{\eth}{ð}
\newcommand{\thorn}{þ}
\newcommand{\wynn}{ƿ}

\begin{document}
\maketitle
\tableofcontents

\chapter{}

\textit{Corpus and Various documents distributed in class}

The primary course book will be "an introduction to middle English".

Middle English (defined as the period between the Norman conquest and the
introduction of the printing press) is handed down to us entirely in
hand-written manuscript of which several copies tend to exist.

Hunter.uma.es

The Hengwrt is the oldest preserved manuscript of the Canterbury
tales\footnote{Known in old manuscripts as ``The book of the tales of
Canterbury''}. The manuscript is written in \textit{Gothic Cursiva Anglicana
Formata}. This script is part of the Gothic family (the exclusive family after
the Norman conquest).  Of this family there are two types, one of which is
\textit{Anglicana}.  The script is written cursively though adapted for
literature (though separation of the letters) (indicated by the \textit{Formata}
in the script name).

Common in old manuscripts is the concept of biting, which is where the two
letters share a side, such as in \ae.

Modern elements of English absent from Middle English include the term ``who''
as a relative pronoun, being substituted instead by ``that''. Similarly,
do-support for verbs has not yet come into being.

\section{Lecture}

\textit{Lecture slides supplied on-paper in class and available online}

English arrives in Britain from the continent (Friesland, Germany, and Saxony)
after the departure of the Romans. At this time, most people speak a Germanic
language close to Welsh, pushing away the indigenous Celts who feel to the
Scottish Ilses, Cornwall and Ireland.

As the viking raids start in Northumbria (793), there is more and more contact
between the English west-Germanic language, and Scandinavian north-Germanic
language. The vikings, other than raiding, likely settled peacefully coexisting
with the Anglo Saxons. Given this extended coexistence, there is a great share
of vocabulary which arises only in manuscripts of Middle English. Old English
manuscripts survive primarily out of areas such as Wessex, which had little
Scandinavian (Danish) influence. Towards the end of this period -- right before
the Norman conquest -- a Norwegian sits on the throne, leaning -- after his
death -- to a dispute between the Norwegians, English, and French.

After the Norman conquest, the official language of England became French with
all government officials, aristocrats, and bishops being French nationals. This
conquest affects English in many ways besides language, such as format,
handwriting and poetry. After this period, when English returns as a written
language, no one knows how to spell it, having written only French and Latin for
300 years, leading to great variation in spelling.

\chapter{Codicology and Paleography}

\textit{Lecture slides supplied on-paper in class and available online}

Middle English (and in fact, all medieval manuscripts) contain inherent
ambiguity due to the difficulty of recognizing handwriting, with manuscripts
containing otiose marks\footnote{Serving no linguistic purpose}, having no
normalized spacing or capitalisation, and frequent use of abbreviation.

A letter can defined by its nomen, figura, and potestas, which together from the
littera.

\begin{tabular}{l|l}
Term & Meaning \\
\hline
	Littera & Letter \\
	Nomen & Name \\
	Figura & Shape \\
	Potestas & Sound (literally ``power'')\\
\end{tabular}


\end{document}
