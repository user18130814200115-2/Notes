\documentclass[12pt]{report}

\usepackage{polyglossia}
\setdefaultlanguage{english}
\setotherlanguage[variant=ancient]{greek}

%\usepackage[style=philosophy-classic, language=english, uniquename=init, mincitenames=2, minbibnames=2]{biblatex}
%\addbibresource{/home/user/.local/share/latex/general.bib}

\title{Ethics}
\date{Autumn 2023}
\author{Wester, T.I.}


\renewcommand{\chaptername}{Lecture}


\begin{document}
\maketitle
\tableofcontents

\chapter{}

Forgiveness is a rather commonsensical idea. On the face of it, forgiveness is
given by a victim to a wrongdoer. Forgiveness is something the wrongdoer can
ask for, and is in some sense connected to the emotions of both the parties
involved.

This simple images becomes complicated in reality however, as the asymmetrical
system of ``victim forgives wrongdoer'' does not always apply. There is the
concept of forgiving oneself, as well as third-party forgiveness (such as
presidential pardons, or Biblical forgiveness).

\chapter{}

There is a distinction to be made between immediate reaction and true
resentment. It is natural -- and unavoidable -- for someone to instinctively
react to injustice before they even think. Resentment follows after a moral
judgement. The taking of revenge occurs only in cases of immediate reaction, or
in cases of failed moral judgements.

\section{christian forgiveness}

Jesus tells us to forgive, and furthermore claims to have forgiven our sins.
However, the question remains hoe this forgiveness works, as there are not
obvious victims for sins other than the sinner themselves.

According to Griswold, forgiveness is not about administration of justice, but
rather changes the relationship between the wrongdoer and victim in a healthier
direction. Forgiveness then is an ethical response to wrongdoing which forswears
revenge.

In the case of Christianity then, the forgiveness granted by Jesus would be more
akin to forgiving one's self.

Being in a state of resentment furthermore makes one vulnerable, thus adding
another Christian angle, as forgiveness makes us less vulnerable and more
God-like.


\end{document}
