\documentclass[12pt]{report}

\usepackage{polyglossia}
\setdefaultlanguage{english}
\setotherlanguage[variant=ancient]{greek}

%\usepackage[style=philosophy-classic, language=english, uniquename=init, mincitenames=2, minbibnames=2]{biblatex}
%\addbibresource{/home/user/.local/share/latex/general.bib}

\title{Ethics}
\date{Autumn 2023}
\author{Wester, T.I.}


\renewcommand{\chaptername}{Lecture}


\begin{document}
\maketitle
\tableofcontents

\chapter{}

Forgiveness is a rather commonsensical idea. On the face of it, forgiveness is
given by a victim to a wrongdoer. Forgiveness is something the wrongdoer can
ask for, and is in some sense connected to the emotions of both the parties
involved.

This simple images becomes complicated in reality however, as the asymmetrical
system of ``victim forgives wrongdoer'' does not always apply. There is the
concept of forgiving oneself, as well as third-party forgiveness (such as
presidential pardons, or Biblical forgiveness).


\end{document}
