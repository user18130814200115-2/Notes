\documentclass[12pt]{article}
\usepackage{../Auxiliary/wester}

\usepackage{polyglossia}
\setdefaultlanguage{english}
\setotherlanguage[variant=ancient]{greek}

%\usepackage[style=philosophy-classic, language=english, uniquename=init, mincitenames=2, minbibnames=2]{biblatex}
%\addbibresource{/home/user/.local/share/latex/general.bib}

\usepackage{amssymb}

\newcommand{\would}{\mathbin{\Box}{\rightarrow}}
\newcommand{\might}{\mathbin{\Diamond}{\rightarrow}}

\title{Philosophy of Language and Logic}
\subtitle{Reading summary\\Monday September 11th 2023}
\date{Monday September 11th, 2023}
\author{Wester, T.I.}

\begin{document}
\maketitle


The first three sections of Lewis' paper on counterfactuals cover roughly three
topics. First is the matter of possible world semantics. Second is the defence
of the claim that the counterfactual cannot be a strict conditional. Third is
the defence of Lewis' analysis of the counterfactual as a variable conditional.

\section{Possible world semantics}

In possible world semantics, a world describes a certain state of things wherein
all truth values are included. Worlds exist in accessibility relations to each
other, every world ($W_n$) wherein the truth values are not impossible in
relation to a starting world ($A$) makes for an accessibility relation between
$A$ and $W_n$. This relation is no necessarily symmetrical. It is generally
taken that $A$ is also in $W$.

Thus, we have for instance the actual world $A$ and a set of worlds $W$ which
are accessible from $A$. Is a certain proposition $\phi$ is true in every world
in $W$, then $\phi$ is necessary, expressed as $\Box \phi$. If $\phi$ holds
true in at least one world in $W$, then $\phi$ is possible, expressed as
$\Diamond \phi$. Of course $\Box \phi \supset \Diamond \phi$.

\subsection{Counterfactual}

Building on the existing notation, Lewis introduces a symbol for two
counterfactuals $\would$ and $\might$ these two indicating respectively sentences
of the structure ``If it were the case that $x$ then it \emph{would} be the case
that $y$'' and  ``If it were the case that $x$ then it \emph{might} be the case
that $y$''.

\section{Strict conditional}

Lewis then goes on to argue that $\would$ can not be a strict conditional.
Strict conditionals are conditionals of the type $\Box(\phi \supset \psi)$.
Meaning that in every world accessible from the current one, the proposition
$\phi \supset \psi$ holds true. This then, Lewis argues is distinctly different
from saying ``If it were the case that $\phi$, then it would be the case that
$\psi$''. The difference between strict conditionals and counterfactuals comes
when we consider several of them together.

Lewis formulates various examples where new information to a given scenario
reverses the outcome. Neither the first implication nor the one with added
information make the counterfactual act unlike a strict conditional, but any two
taken together do.

I have one problem with Lewis' formulation, namely that his examples don't
strictly hold true. ``If you walk on the lawn, the lawn will be fine, but
everyone did, the lawn would be ruined''. The first part of this sentence is not
true, because it assumes the information in the second one. As with most logical
implicatures, there is a lot of background information that is silently assumed.
So too for the party example, ``If Otto had come, it would have been a lively
party'' says Lewis, but Otto came, and the party was not lively, because Otto's
nemesis Anna was also present. The first statement is simply not true. If there
is a party with a number of attendants, then the statement can be made that the
party would have been lively \emph{if only} Otto had come. But if the guest list
is unknown, the statement is simply false.

Lewis then explains how the strictness of the strict conditional restricts it to
certain accessibility relationships with possible worlds -- or, as he calls
them, spheres of influence. Taking the examples he gave before, he says that
there exist no level of strictness wherein each statement is true on its own
(with or without extraneous informations), but wherein they nevertheless do not
contradict each other.

\section{Variably strict conditionals}

Having concluded that there is not level of strictness that allows the
counterfactual to operate as a strict conditional, Lewis moves on to stating and
defending his own point of view; that counterfactuals operate as variably strict
conditionals. This allows for the existence of a set of spheres of influence for
the examples above, wherein the sphere is such that it allows for truth in one
case, whist just missing out on the extra information which would falsify the
statement in a different sphere.

In other words, the accessibility relations between words are such that new ones
are added alongside the new information so as to change the possibilities.


\end{document}
