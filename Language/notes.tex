\documentclass[12pt]{report}

\usepackage{polyglossia}
\setdefaultlanguage{english}
\setotherlanguage[variant=ancient]{greek}

%\usepackage[style=philosophy-classic, language=english, uniquename=init, mincitenames=2, minbibnames=2]{biblatex}
%\addbibresource{/home/user/.local/share/latex/general.bib}

\title{Philosophy of Language and Logic}
\date{Autumn 2023}
\author{Wester, T.I.}


\renewcommand{\chaptername}{Lecture}


\begin{document}
\maketitle
\tableofcontents

\chapter{Introduction to conditionals}

Conditionals in the English language exist in various constructions and operate
in equally various ways. Categorizing these constructions and unifying them
under one theory is a seemingly transparent question, though it turns out more
complicated than one might anticipate.  Given a number of conditional
statements, one might conclude the connecting factor to be the word \emph{``if''},
or an equivalent term in other languages. This word will turn out to play a
major part in the analysis of conditionals, though it is not a proper indicator,
as it is neither necessary nor necessarily indicative.

While the topic of conditionals is not simple, it is definitely worthwhile
studying, as conditionals are used in much of our thinking, argumentation, and
by-extension: philosophising. Within philosophy, conditionals are used to
express various concepts such as dispositions and causation as well as being
used for basic philosophical Logic. Due to this wide use, it is nearly
impossible to study philosophy of language without encountering the problem of
conditionals.

\section{Kinds of conditionals}

We can divide conditionals broadly into at least two categories. There are those
sentences that indicate a state of affairs and those that indicate a
possibility. These are respectively called \emph{indicative} and
\emph{subjunctive}\footnote{These are also occasionally called
\emph{counterfactual}}.

Indicative conditionals relate to the material conditional ($\supset$) of
classical logic. Whereas the subjunctive conditionals do not. In fact -- they do
not relate to any concept in classical logic, and therefore lack a
straightforward method of analysis, they are not truth-functional.

Despite the obvious connection between the indicative conditional and material
conditional, they are not necessarily the same, and a large body of literature
is written on the topic.  For this reason, we cannot express the material
conditional in English by using the typical construction ``if ... then'' as it
would confuse the material conditional with the indicative. Therefore, we
express the material conditional instead using a different operator, namely
\emph{or} ($\lor$). Thus, $A \supset C$ becomes $\lnot A \lor C$.

Subjunctive conditionals can further be separated into \emph{would subjunctives}
and \emph{might subjunctives}. These indicate the words used in the respective
sentences which relate to whether they express a possible consequent or a
definitive one, though in either case the antecedent in negated (thus the term
``counterfactual''). This course will hardly ever mention the might subjunctive,
and any reference to ``the subjunctive'', unless otherwise noted, can be taken
to refer to the would subjunctive.

These various conditionals will be formalized using the following symbols:
\begin{tabular}{l|l}
	Conditional & symbol\\
	\hline
	Material & $\supset$\\
	Indicative & $\rightarrow$\\
	Subjunctive & $>$\\
	Might & $>_m$\\
\end{tabular}

\part{Indicatives}
\chapter{The material analysis}

``The material analysis'' is a general term for analysing the indicative
conditional ($\rightarrow$) as being logically equivalent to the material
conditional ($\supset$)\footnote{This is also sometimes called ``the horseshoe
analysis'', referring to the shape of the material conditional symbol
($\supset$).}. The difficulty in proving the material analysis comes in the fact
that the indicative conditional cannot be assumed to be truth-functional, thus
not allowing a mere truth-functional analysis.

If the indicative conditional is truth-functional, it must be equivalent to the
material conditional. This belief stems from the following proof:
\begin{enumerate}
	\item
		$P \land Q \rightarrow P$ is assumed to be a logical truth 
	\item
		$A \rightarrow C$ is assumed not to be a logical truth
\end{enumerate}

Given these assumptions, if $\rightarrow$ is truth functional, it must function
in the same way in every case. Assuming that (1) is a logical truth allows us to
derive the following three truths:
\begin{itemize}
	\item
		$F \rightarrow T$
	\item
		$T \rightarrow T$
	\item
		$F \rightarrow F$
\end{itemize}
Given (2), and the derived truths, we can also know derive one falsehood:
\begin{itemize}
	\item
		$T \rightarrow F$
\end{itemize}
Which leaves us with the truth table for the material conditional.

The reason this discussion remains interesting is because (1) and (2) are
assumed, rather than argued for.

\section{Argument for the material analysis}

One reason for accepting the material analysis is the following: Given two
mutually exclusive things, the indicative cannot be true. Similarly, the
material conditional cannot be true. Meaning that the material conditional
\emph{entails} the indicative.

More interesting is the other direction (the indicative conditional entailing
the material conditional). This is called the ``or to if'' analysis which goes
as follows: Given two options, if we know that one option is not the case, it
implies that the other is. This is a widely held assumption which leads us to
accept that the indicative \emph{entails} the material conditional.

Given that the material conditional and indicative entail each other, they are
logically equivalent.

\section{Paradoxes}

Despite the aforementioned ``proof'', there are cases where we encounter
paradoxes due to the fact that any false antecedent leads to a true statement in
the medial conditional. Or the fact that a true consequent, leads to a true
statement. While these are logically coherent truths, the ability to put
\emph{any} consequent or antecedent, whether nonsensical, related or otherwise,
allows us to prove anything we wish.

\section{Implicatures}

If something is ``implied'', it is not stated directly, but nevertheless
conveyed. It may be said that $A \rightarrow C$ \emph{states} only $A \supset
C$, while \emph{implying} more. Such non-direct statements make problems for the
paradoxes listed above.

Grice distinguishes between conversational-, and conventional-implicates, which
together remove the paradoxes from the material analysis.

\subsection{Conversational Implicatures}

In a conversation, we implicitly follow a certain principle know as ``the
cooperative principle'' and we are allowed to assume that our interlocutors
follow this principle. The principle can be generalized as a set of rules such
as:
\begin{enumerate}
	\item
		Say no more or less than appropriate
	\item
		Make true contributions
	\item
		Be relevant to the topic being discussed
	\item
		etc.
\end{enumerate}

There are various -- intentional and unintentional -- ways in which we violate
these rues where we imply certain information while staying cooperative.

\chapter{Non-material analyses}

\part{Subjunctives}
\chapter{Similarity semantics}

\chapter{Non-similarity semantics}

\end{document}
