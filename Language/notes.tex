\documentclass[12pt]{report}

\usepackage{polyglossia}
\setdefaultlanguage{english}
\setotherlanguage[variant=ancient]{greek}

%\usepackage[style=philosophy-classic, language=english, uniquename=init, mincitenames=2, minbibnames=2]{biblatex}
%\addbibresource{/home/user/.local/share/latex/general.bib}

\title{Philosophy of Language and Logic}
\date{Autumn 2023}
\author{Wester, T.I.}


\renewcommand{\chaptername}{Lecture}


\begin{document}
\maketitle
\tableofcontents

\chapter{Introduction to conditionals}

Conditionals in the English language exist in various constructions and operate
in equally various ways. Categorizing these constructions and unifying them
under one theory is a seemingly transparent question, though it turns out more
complicated than one might anticipate.  Given a number of conditional
statements, one might conclude the connecting factor to be the word \emph{``if''},
or an equivalent term in other languages. This word will turn out to play a
major part in the analysis of conditionals, though it is not a proper indicator,
as it is neither necessary nor necessarily indicative.

While the topic of conditionals is not simple, it is definitely worthwhile
studying, as conditionals are used in much of our thinking, argumentation, and
by-extension: philosophising. Within philosophy, conditionals are used to
express various concepts such as dispositions and causation as well as being
used for basic philosophical Logic. Due to this wide use, it is nearly
impossible to study philosophy of language without encountering the problem of
conditionals.

\section{Kinds of conditionals}

We can divide conditionals broadly into at least two categories. There are those
sentences that indicate a state of affairs and those that indicate a
possibility. These are respectively called \emph{indicative} and
\emph{subjunctive}\footnote{These are also occasionally called
\emph{counterfactual}}.

Indicative conditionals relate to the material conditional ($\supset$) of
classical logic. Whereas the subjunctive conditionals do not. In fact -- they do
not relate to any concept in classical logic, and therefore lack a
straightforward method of analysis, they are not truth-functional.

Despite the obvious connection between the indicative conditional and material
conditional, they are not necessarily the same, and a large body of literature
is written on the topic.  For this reason, we cannot express the material
conditional in English by using the typical construction ``if ... then'' as it
would confuse the material conditional with the indicative. Therefore, we
express the material conditional instead using a different operator, namely
\emph{or} ($\lor$). Thus, $A \supset C$ becomes $\lnot A \lor C$.

Subjunctive conditionals can further be separated into \emph{would subjunctives}
and \emph{might subjunctives}. These indicate the words used in the respective
sentences which relate to whether they express a possible consequent or a
definitive one, though in either case the antecedent in negated (thus the term
``counterfactual''). This course will hardly ever mention the might subjunctive,
and any reference to ``the subjunctive'', unless otherwise noted, can be taken
to refer to the would subjunctive.

These various conditionals will be formalized using the following symbols:
\begin{tabular}{l|l}
	Conditional & symbol\\
	\hline
	Material & $\supset$\\
	Indicative & $\rightarrow$\\
	Subjunctive & $>$\\
	Might & $>_m$\\
\end{tabular}

\part{Indicatives}
\chapter{The material analysis}




\chapter{Non-material analyses}

\part{Subjunctives}
\chapter{Similarity semantics}

\chapter{Non-similarity semantics}

\end{document}
