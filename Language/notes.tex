\documentclass[12pt]{report}

\usepackage{polyglossia}
\setdefaultlanguage{english}
\setotherlanguage[variant=ancient]{greek}

%\usepackage[style=philosophy-classic, language=english, uniquename=init, mincitenames=2, minbibnames=2]{biblatex}
%\addbibresource{/home/user/.local/share/latex/general.bib}

\title{Philosophy of Language and Logic}
\date{Autumn 2023}
\author{Wester, T.I.}

\renewcommand{\chaptername}{Lecture}

\usepackage{amssymb}
\newcommand{\would}{{\mathbin{\Box}{\rightarrow}}}
\newcommand{\might}{{\mathbin{\Diamond}{\rightarrow}}}


\begin{document}
\maketitle
\tableofcontents

\chapter{Introduction to conditionals}

Conditionals in the English language exist in various constructions and operate
in equally various ways. Categorizing these constructions and unifying them
under one theory is a seemingly transparent question, though it turns out more
complicated than one might anticipate.  Given a number of conditional
statements, one might conclude the connecting factor to be the word \emph{``if''},
or an equivalent term in other languages. This word will turn out to play a
major part in the analysis of conditionals, though it is not a proper indicator,
as it is neither necessary nor necessarily indicative.

While the topic of conditionals is not simple, it is definitely worthwhile
studying, as conditionals are used in much of our thinking, argumentation, and
by-extension: philosophising. Within philosophy, conditionals are used to
express various concepts such as dispositions and causation as well as being
used for basic philosophical Logic. Due to this wide use, it is nearly
impossible to study philosophy of language without encountering the problem of
conditionals.

\section{Kinds of conditionals}

We can divide conditionals broadly into at least two categories. There are those
sentences that indicate a state of affairs and those that indicate a
possibility. These are respectively called \emph{indicative} and
\emph{subjunctive}\footnote{These are also occasionally called
\emph{counterfactual}}.

Indicative conditionals relate to the material conditional ($\supset$) of
classical logic. Whereas the subjunctive conditionals do not. In fact -- they do
not relate to any concept in classical logic, and therefore lack a
straightforward method of analysis, they are not truth-functional.

Despite the obvious connection between the indicative conditional and material
conditional, they are not necessarily the same, and a large body of literature
is written on the topic.  For this reason, we cannot express the material
conditional in English by using the typical construction ``if ... then'' as it
would confuse the material conditional with the indicative. Therefore, we
express the material conditional instead using a different operator, namely
\emph{or} ($\lor$). Thus, $A \supset C$ becomes $\lnot A \lor C$.

Subjunctive conditionals can further be separated into \emph{would subjunctives}
and \emph{might subjunctives}. These indicate the words used in the respective
sentences which relate to whether they express a possible consequent or a
definitive one, though in either case the antecedent in negated (thus the term
``counterfactual''). This course will hardly ever mention the might subjunctive,
and any reference to ``the subjunctive'', unless otherwise noted, can be taken
to refer to the would subjunctive.

These various conditionals will be formalized using the following symbols:
\begin{tabular}{l|l}
	Conditional & symbol\\
	\hline
	Material & $\supset$\\
	Indicative & $\rightarrow$\\
	Subjunctive & $>$\\
	Might & $>_m$\\
	Subjunctive (alternative) & $\would$\\
	Might (alternative) & $\might$\\
\end{tabular}

\part{Indicatives}
\chapter{The material analysis}

``The material analysis'' is a general term for analysing the indicative
conditional ($\rightarrow$) as being logically equivalent to the material
conditional ($\supset$)\footnote{This is also sometimes called ``the horseshoe
analysis'', referring to the shape of the material conditional symbol
($\supset$).}. The difficulty in proving the material analysis comes in the fact
that the indicative conditional cannot be assumed to be truth-functional, thus
not allowing a mere truth-functional analysis.

If the indicative conditional is truth-functional, it must be equivalent to the
material conditional. This belief stems from the following proof:
\begin{enumerate}
	\item
		$P \land Q \rightarrow P$ is assumed to be a logical truth 
	\item
		$A \rightarrow C$ is assumed not to be a logical truth
\end{enumerate}

Given these assumptions, if $\rightarrow$ is truth functional, it must function
in the same way in every case. Assuming that (1) is a logical truth allows us to
derive the following three truths:
\begin{itemize}
	\item
		$F \rightarrow T$
	\item
		$T \rightarrow T$
	\item
		$F \rightarrow F$
\end{itemize}
Given (2), and the derived truths, we can also know derive one falsehood:
\begin{itemize}
	\item
		$T \rightarrow F$
\end{itemize}
Which leaves us with the truth table for the material conditional.

The reason this discussion remains interesting is because (1) and (2) are
assumed, rather than argued for.

\section{Argument for the material analysis}

One reason for accepting the material analysis is the following: Given two
mutually exclusive things, the indicative cannot be true. Similarly, the
material conditional cannot be true. Meaning that the material conditional
\emph{entails} the indicative.

More interesting is the other direction (the indicative conditional entailing
the material conditional). This is called the ``or to if'' analysis which goes
as follows: Given two options, if we know that one option is not the case, it
implies that the other is. This is a widely held assumption which leads us to
accept that the indicative \emph{entails} the material conditional.

Given that the material conditional and indicative entail each other, they are
logically equivalent.

\section{Paradoxes}

Despite the aforementioned ``proof'', there are cases where we encounter
paradoxes due to the fact that any false antecedent leads to a true statement in
the medial conditional. Or the fact that a true consequent, leads to a true
statement. While these are logically coherent truths, the ability to put
\emph{any} consequent or antecedent, whether nonsensical, related or otherwise,
allows us to prove anything we wish.

\section{Implicatures}

If something is ``implied'', it is not stated directly, but nevertheless
conveyed. It may be said that $A \rightarrow C$ \emph{states} only $A \supset
C$, while \emph{implying} more. Such non-direct statements make problems for the
paradoxes listed above.

Grice distinguishes between conversational-, and conventional-implicates, which
together remove the paradoxes from the material analysis.

\subsection{Conversational Implicatures}

In a conversation, we implicitly follow a certain principle know as ``the
cooperative principle'' and we are allowed to assume that our interlocutors
follow this principle. The principle can be generalized as a set of rules such
as:
\begin{enumerate}
	\item
		Be appropriate in amount
	\item
		Make true contributions
	\item
		Be relevant to the topic being discussed
	\item
		etc.
\end{enumerate}
There are various -- intentional and unintentional -- ways in which we violate
these rues where we imply certain information while staying cooperative.

With the cooperative principle, we can argue against some of the paradoxes
listed in the earlier section, and -- by extension -- argue for the material
analysis. This would be because, while an implication is logically true,
falsehood is implied by the context of the conversation.

\subsection{Conventional Implicatures}

Conventional implicatures are tied to the manner in which we use particular
words, where a certain word in a given context implies a different meaning than
the apparent one.

As such, we can say that $A \rightarrow C$ implicates \emph{a connection
between} $A$ and $C$. Thus, given examples wherein the antecedent and consequent
are not in any way related to each other, we go against the conventional use of
$\rightarrow$.

\subsection{Criticism}

While conventional-, and conversational- implicatures solves the paradoxes in
the material analysis. It may be said that the implied information of a given
word (in the case of conventional implicatures) is simply truth-functional.
Meaning that the difference between $\supset$ and $\rightarrow$ is a logical one
rather than a implied difference, thus causing a problem for the material
analysis.

Furthermore, conventional implicatures may be said to contribute something to the
\emph{tone} of a statement, changing not the meaning, but the way of stating the
meaning. This does not apply to $\rightarrow$ however, where the paradoxes are
not problems in tone, but distinctly in meaning.

\chapter{Non-material analyses}

\section{Probability theory}

In probability theory, there exist two main notions of the mathematics of
probability, and by extension philosophical probability. These two notions are
the subjective-, and objective- notions, referring respectively to the epistemic
perspective of a given agent, and the agent-independent features of the world.
Subjective probability is something we often encounter, for instance due to
ignorance of all the facts. Object probability can be seen for instance in
quantum events, and prior to the discovery of quantum theory, it was doubted
that objective probabilities existed\footnote{For instance Einstein: ``God does
not play dice''}.

\begin{center}
	\textit{Axioms of probability theory handed out in class}
\end{center}

$P(A|B) = \frac{P(A) \cdot P(B|A)}{P(B)}$, all the rest is commentary

\subsection{Notation}
\begin{tabular}{l|l}
Symbol & Meaning \\
\hline
$P(Q)$ & Probability of $Q$\\
$P(R|Q)$ & The probability of $Q$ \emph{given} $R$\\
$P_A$ & The updated function given new information $A$\\
\end{tabular}

\section{Conditional probability}

According to the Ramsey test, $P(A \rightarrow C) = P_A(C) and threrefor
P(C|A)$. In other words, the probability of an indicative is the same as the
probability of the consequent given the antecedent is true. The proof for this
is given by entering $P(Q \rightarrow R)$ in AT, followed by the ratio formula,
and the if-and conversion, simplifying by T1 and T3\footnote{See handout for
reference numbers}.

However, if There is a chance of that $C$ is true or false, given $A$, then $P(A
\rightarrow C) = P(R)$, according to the axioms. This result is obviously not
true though, as various counterexamples exist.  The same problem holds for
combining the Ramsey test and the ratio formula, which gets us $P(R \and Q) =
P(R) \cdot P(Q)$, which allows for more mathematical counterexamples.

Both of these formulae from problems for the Ramsey test, as the axioms of
probability theory are not to be doubted.

\section{Appiah}

Appiah argues against Jackson's material analysis as expounded in the previous
chapter. He does this though formulating an embedded conditional which is
acceptable due to various consequences of probability theory. If Jackson goes
along with the fact that robustness can be expressed in terms of robustness, and
furthermore allows for embedded conditionals, then Appiah's argument is
convincing. Furthermore, Appiah requires probability to be dependent only on what
one is saying, not on what one is implying\footnote{potential paper topic}.
A further consideration is the definition of the term ``high probability'', as
one may claim this to also be context dependent.

\begin{center}
\textit{See slides ``Rebutting Robustness I''}
\end{center}

\part{Subjunctives}
\chapter{Similarity semantics}

TO talk about subjunctives, propositional logic needs to be expanded with modal
operators. Modal logic speaks about necessary and possible things, represented
respectively by $\Box$ and $\Diamond$, either of which can be taken as a
primitive for modal logic. Crucially, $\Box$ and $\Diamond$ are not truth
functional, as for any $p$, $\Box p$ is unsolvable if $p$ is true, and false if
$p$ is false.

\begin{tabular}{l|l}
$p$ & $\Box P$ \\
\hline
T & \\
F & F\\
\end{tabular}

To analyse modality then, we use the method of possible worlds. Something is
necessary if it is the case in every possible world, whereas it is possible if
true in at least one world. In doing this, we limit ourselves to certain worlds
as described by an accessibility relation. 

Strict conditionals are a type of material conditional where the implication is
taken as necessary $\Box(A \supset C)$. This is true if $A \supset C$ is true in
every accessible world.

The subjunctive $A \would C$ is sometimes analysed as being equivalent to $\Box
(A \supset C)$ This makes the subjunctive a strict conditional. Lewis argues
against this idea. 

He does this through Antecedent Strengthening. This is a structure where, in any
given world, $\Box(A \supset C)$ is true and $\Box A \land B \supset C$ is also
true by extension. However, for subjunctives, there are many examples where $B
\supset \lnot C$.

Lewis thus concludes that $\would$ is not a strict conditional, it is a variably
strict conditional, where the accessibility of worlds id dependent on the
antecedent. This is called similarity semantics, as only those worlds are
considered that are more similar to the world of evaluation.

To check the truth value of a subjunctive conditional under similarity
semantics, we must look at the closest worlds where the antecedent is true. The
closest here means that we do not look at worlds that are any more different
from the actual world than necessary. Of course, any subjunctive can be made
false or true depending on the accessibility relations and the truth values of
the consequent in every accessible antecedent-world.

One matter under discussion is weather multiple worlds can be equally similar or
dissimilar to a given reference world. According to Lewis: $w$, where $A \land
B$, is dissimilar in equal amounts to $w'$, where $A \land \lnot B$ and $w^*$,
where $\lnot A \land B$. According to Stalnaker, there is a priority of
propositions, where -- for instance -- $A$ being different is more substantial
than $B$ being different, thus placing $w'$ and $w^*$ exist in different
spheres. Stalnaker's theory leaves the question of \emph{why} this priority
exists.

Stalnaker appeals to the vague nature of subjunctives, thus making it that there
is not a single correct function which selects the closest antecedent world. A
subjunctive in Stalnaker's theory then is true according to every possible
selection function. This is called supervaluation, and it allows for the
benefits of Lewis' theory while still only allowing one most similar world
\emph{per selection function}. In problematic cases, one may then speak of truth
as being truth in all possible worlds under a specific selection function, while
we speak of super-truth if the proposition is true under every relevant
selection function.

Lewis believes that $\would$ and $\might$ stand in a simmilar relationship as
$\Box$ and $\Diamond$ in that $\Diamond p \equiv \lnot \Box \lnot p$ and $a
\might c \equiv \lnot (a \would \lnot c)$. This allows us to get truth values
for ``might'' sentences, as we can now translate them as ``would sentences''.

This is not possible in Stalnaker's account, as it requires a singular closest
world, making $\would$ and $\might$ the same. Thus, Stalnaker gives a different
account of $\might$

\chapter{Non-similarity semantics}

Lewis's system of possible words, wherein other worlds exists but are
inaccessible to us. A world in this system is a maximally complete space-time
continuum, just like the actual world. Nearly everything in logic can be reduced
to possible world semantics. Propositions for instance are merely the set of
possible worlds wherein that proposition holds true. This however requires us to
accept that things which do not exist in the actual world truly exist in other
worlds, as long as those things are possible. Thus, flying donkeys exist, they
are just inaccessible to us. This system is called extreme realism and any
philosophers are not willing to accept it.

Am alternative system is abstract realism, where a world is viewed as a set of
sentences. A world is thus identified as a proposition which describes
completely a way for things to be. This proposal is usually not reductive, as
extreme realism tends to be.

\section{Sobel sequences}

Lewis considered cases wherein an implicative is true, though false in light of
new information $A \would C$ and $A \land B \would \lnot C$ . These sequences
are called Sobel sequences, and are explained in a previous chapter. Lewis uses
this problem to argue for his similarity semantics. Alternatively, other
philosophers argue that both ``necessities'' in the two sentences, rely on
context to receive their meaning. In other words, the epistemic use of modal
logic depends on the epistemic state of the agent.

In allowing this caveat, it allows one to analyse $A \would C$ as $\Box (A
\supset C)$ and  $A \land B \would \lnot C$ as $\Box (A \land B \supset \lnot
C)$ while granting that the $\Box$ in the former is not equivalent to the $\Box$
in the latter, thus avoiding the seeming logical inconsistency. Ultimately, the
result is the same as Lewis' theory. However, Lewis' theory allows for a more
systematic analysis than in possible for mere context dependant proposals. One
can hover give a systematic account of the context dependence of conditionals.

One such systematic account is given by Fintel and Gillies. They argue that the
context dependent approach is more powerful than Lewis' account, through using
reverse Sobel sequences. It is claimed by Fintel and Gillies that these
sequences are logically incoherent. This incoherence cannot be explained by
Lewis, as his theory does not allow the context (such as order) to influence
truth-value. The context for Fintel and Gillies determines the accessibility
relation, and is therefore both truth-functional and systematic.

In the Sobel sequences listed above then, $B$ is not so much influence the
truth, but rather the context. Uttering $B$ shifts the context so as to change
the evaluation of $A \would C$. Of course, $A$ is also context function, as the
utterance $A \rightarrow$ already establishes the context which mks the
accessibility relation such that there are at least some A-worlds accessible.

\section{Epistemic Irresponsibility}

Moss argues that similarity semantics can deal with reverse Sobel sequences,
arguing from a pragmatic perspective. She proposes a principle which states that
it is epistemicaly irresponsible to assert a proposition which is incompatible
with a salient possibility which cannot be rules out. In other words, we cannot
assert $C$ given $A$ in $A \would C$ because we cannot rule out $B$ and $A \land
B \would \lnot C$. In the case of reverse Sobel sequences, there is no salient
possibility, as the most complete statement is the final one in the Sobel
sequence.

This principle also appeals to context, though contrary to Fintel and Gillies'
account, epistemic irresponsibility does not require the dynamic changing to
contexts.

Epistemic Irresponsibility works for many -- but not all -- cases. There are
cases where the first statement does not make a salient possibility, because a
statement which is phrased like a possibility is too outlandish to be salient.
In these cases, epistemic irresponsibility does not apply.

\part{Online Lectures}
\chapter{Stalnaker on Indicatives}

Previously in this course we have discussed Stalnaker's theory on subjunctives,
where $A \might C$ holds if $C$ is true in the closest world where $A$ is true.
He uses a similar theory for Indicatives. See the section on Indicatives for a
more in-depth explanation of Stalnaker's views.

In evaluating a indicative conditional, there is a constraint on the function
which selects the worlds available for discussion. This constraint is
contextual, and limits -- for instance -- to worlds wherein common knowledge is
available. This limit makes it so that we cannot assert a consequent which has
nothing to do with the antecedent, as the selection function limits us to worlds
wherein the antecedent is relevant to the consequent.

This context limitation leads to two constraints. The first being that we cannot
asset $A \rightarrow C$ without there being at least one world where $A$ holds
in the context set. Likewise, we cannot asset $A \lor B$ without there being
both a wold where $A$ holds and not $B$, and a world where $A$ does not hold and
$B$ does. This pragmatic constraint indicates the conventional reading of ``or''
as exclusive.

Stalnaker makes a distinction between \textit{reasonable inferences} and
\textit{valid inferences} (the latter one also called entailment). As such, he
claims that the material analysis is reasonable but not valid. An inference from
P to Q is reasonable if everyone committed to P is also committed to Q. The same
inference is valid if P cannot be true without Q being true as well. As such, a
valid inference is essential, while reasonable inference is accidental.




\end{document}
