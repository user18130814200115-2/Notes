\documentclass[12pt]{article}
\usepackage{../Auxiliary/wester}

\usepackage{polyglossia}
\setdefaultlanguage{english}
\setotherlanguage[variant=ancient]{greek}

%\usepackage[style=philosophy-classic, language=english, uniquename=init, mincitenames=2, minbibnames=2]{biblatex}
%\addbibresource{/home/user/.local/share/latex/general.bib}

\title{Philosophy of Language and Logic}
\subtitle{Reading summary\\Monday September 4th 2023}
\date{Monday September 4th, 2023}
\author{Wester, T.I.}

\begin{document}
\maketitle

Jackson argues in favour of the indicative conditional\footnote{ Marked with a
$\rightarrow$} being logically equivalent to the material
conditional\footnote{Marked with a $\supset$}. This analysis has some merits,
though it is not without issues. There are various counterexamples to the
so-called ``material analysis'' which Jackson discusses in his work. He argues
that these counterexamples do not form a problem for the material analysis due
to the fact that these examples do not adhere to certain principle of
robustness. A property $P$ can be said to be robust in regards to $E$, if one
would believe $P$ even if one came to know $E$. For instance ``$P$ or anyways
$E$'' expresses the idea that ``$P \lor E$'' robustly with respect to $\neg P$

\begin{center}
	\textit{Thus far the content of the previous class}
\end{center}
Jackson's theory about the conventional-, and conversational- markers is also
not without problems though, as is pointed out in this week's reading by Anthony
Appiah. Appiah raises a total of four objections which I shall lay out below.

\section{Modus Ponens}

Appiah's first point concerns the nature of modus ponens logic in the English
language. His claim is that, while robustness is an obvious feature of the
English language -- being expressible in certain contexts such as but-sentences
-- it is not so obvious in the case of the indicative. Rather, when it comes to
the indicative, the evidence E (in light of which we would still believe
property P) is not a simple English sentence, but is often scattered across a
statement. This forms a problem for the material analysis due to the value
Appiah ascribes to simplicity, which he expresses in his closing statement:
``And if that is an important fact about me, why shouldn't our language have a
short and simple way of expressing it?''

\section{Other conventional implicatures}

Appiah's next point concerns the nature of the truth-function of ``but'' in
sentences such as ``$A$ but $B$'' and likewise, ``$A$ or anyway $B$''. In these
two cases, there are obvious truth conditions. We use ``but'' if we hold the
following three beliefs:
\begin{enumerate}
	\item
		We believe $A$
	\item
		We believe $B$
	\item
		We believe that there is some kind of contrast between $A$ and
		$B$
\end{enumerate}
This is obvious to speakers of English, though no such obvious observation can
be made for $\rightarrow$. Jackson's claim that $\rightarrow$ can be explained
as $\supset$ is not at all obvious, and furthermore, the functioning of
$\supset$ is not at all obvious.

Appiah then states that there is a commonsensical way to express the truth
function of an English sentence ``If $A$ then $B$'' without making reference to
the material conditional $\supset$.

\section{Robustness}

Next, Appiah goes after Jackson's claim of robustness directly, outside the
scope of conditionals. He does this by addressing the aforementioned case of
``but'' and ``or anyway''. He states that the only \emph{belief} expressed by a
statement ``$A$ but $B$'' is ``$A \land B$''. But that does not entail that
``$A$ \land $B$'' is the truth condition of the sentence ``$A$ but $B$''. Aside
from the \emph{belief}, the sentence also expresses the agent's attitude towards
$A$ and $B$.  This attitude is lost when one uses Jackson's framework of
robustness.

\section{Counterexample}

In his fourth objection, Appiah, more so than raise a concern, formulates a
counterexample to Jackson's theory. His counterexample concerns two parents and
their nursing infant. The structure of their family is such that one of the
parents must always be with the baby. Appiah then uses this to formulate the
following two statements:

\begin{enumerate}
	\item
		``If John comes, Mary won't''
	\item
		``If John comes, the baby won't''
\end{enumerate}

Given the context, both (1) and (2) are likely, but they are not entirely
certain. The only certainty in this situation is the following:
\begin{enumerate}
	\item[3.]
		``If both John and Mary come, so will the baby''
\end{enumerate}
Furthermore, while statement (2) is probable given the context, there is no
actual connection between John's appearance and the appearance of the baby.
Rather, this is a reformulation of (1), where extra context is included, namely
that the baby is likely staying with Mary.

A further point of context is given through the fact that the baby is an infant,
and therefore still breastfeeding, thus leading Mary to not attend longer
gatherings, such as the one under discussion. In other words, statement (1) may
better be put as:
\begin{enumerate}
	\item[4.]
		``Mary won't come''
\end{enumerate}
Formulating it as a conditional is not
entirely accurate, as John's appearance has nothing to do with Mary's.

Appiah uses (1) and (2) to formulate the following statement
\begin{enumerate}
	\item[5.]
		``If John comes, then if Mary comes, the baby won't''
\end{enumerate}
Since this statement is nonsensical given the situation, Appiah concludes that
there must be something wrong with Jackson's theory. Given the worries I raised
in the preceding paragraphs, i would agree with Appiah, provided this is in fact
an accurate application of Jackson's theory.
\end{document}
