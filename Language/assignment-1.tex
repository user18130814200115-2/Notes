\documentclass[12pt]{article}
\usepackage{../Auxiliary/wester}

\usepackage{polyglossia}
\setdefaultlanguage{english}
\setotherlanguage[variant=ancient]{greek}

%\usepackage[style=philosophy-classic, language=english, uniquename=init, mincitenames=2, minbibnames=2]{biblatex}
%\addbibresource{/home/user/.local/share/latex/general.bib}

\title{Philosophy of Language and Logic}
\subtitle{Reading summary\\Monday September 4th 2023}
\date{Monday September 4th, 2023}
\author{Wester, T.I.}

\begin{document}
\maketitle

Jackson argues in favour of the indicative conditional\footnote{in the course
marked with a $\rightarrow$} being logically equivalent to the material
conditional\footnote{In the course marked as a $\supset$}. This analysis has
some merits, though it is not without issues. There are various counterexamples
to the so-called ``material analysis'' which Jackson discusses in his work. He
argues that these counterexamples do not form a problem for the material
analysis due to the fact that these examples do not adhere to certain
principle of robustness. A property P can be said to be robust in regards to E,
if one would believe P even if one came to know E.

\textit{Thus far the content of the previous class}

Jackson's theory about the conventional-, and conversational- markers is also
not without problems though, as is pointed out in this week's reading by Anthony
Appiah. Appiah raises a total of four objections which I shall lay out below.

\section{Modus Ponens}

Appiah's first point concerns the nature of modus ponens logic in the English
language. His claim is that, while robustness is an obvious feature of the
English language -- being expressible in certain contexts such as but-sentences
-- it is not so obvious in the case of the indicative. Rather, when it comes to
the indicative, the evidence E (in light of which we would still believe
property P) is not a simple English sentence, but is often scattered across a
statement. This forms a problem for the material analysis due to the value
Appiah ascribes to simplicity, which he expresses in his closing statement:
``And if that is an important fact about me, why shouldn't our language have a
short and simple way of expressing it?''

\section{}

\section{}

\section{}

In his fourth objection, Appiah, more so than raise a concern, formulates a
counterexample to Jackson's theory. His counterexample concerns two parents and
their nursing infant. 
\end{document}
